%!TeX program = xelatex
\documentclass[12pt,hyperref,a4paper,UTF8]{ctexart}
\usepackage{zjureport}
\usepackage{tikz}
\usepackage{pgfplots}
\usepackage{tabularx}
\usepackage{float}
\definecolor{cmdbg}{rgb}{0.92,0.92,0.92}

%%-------------------------------正文开始---------------------------%%
\begin{document}

%%-----------------------封面--------------------%%
\cover


\thispagestyle{empty} % 首页不显示页码

%%--------------------------目录页------------------------%%
\newpage
\tableofcontents

%%------------------------正文页从这里开始-------------------%
\newpage
本次作业使用三种建模技术来对车的形状进行建模,分别是 Bézier 曲面,Polygonal Mesh 和 L-System。下面分别进行阐述
\section{Bézier 曲面}
Bézier 曲面是由 Bézier 曲线推广而来的,它是由一个或多个 Bézier 曲线组成的。由二维数组描述其控制点,每个控制点是一个三维坐标。Bézier 曲面的方程如下:
\begin{equation}
    \begin{aligned}
        \mathbf{B}(u, v) = \sum_{i=0}^{n} \sum_{j=0}^{m} \mathbf{P}_{ij} B_{i}^{n}(u) B_{j}^{m}(v)
    \end{aligned}
\end{equation}
在 opengl 中,我们按如下方式实现了 Bézier 曲面:

\begin{minted}[    
    linenos,
    breaklines,
    bgcolor = cmdbg,
    fontsize=\footnotesize,
    frame=lines
]{cpp}
void draw_Bézier(){
    GLfloat ctrlpoints[4][4][3] = {
        {{-3, -1.5, 0.0}, {-0.5, -1.5, 0.0}, {0.5, -1.5, 0.0}, {3, -1.5, 0.0}},
        {{-3, -0.5, -2.0}, {-0.5, -0.5, -2.0}, {0.5, -0.5, -2.0}, {3, -0.5, -2.0}},
        {{-3, 0.5, -2.0}, {-0.5, 0.5, -2.0}, {0.5, 0.5, -2.0}, {3, 0.5, -2.0}},
        {{-3, 1.5, 0.0}, {-0.5, 1.5, 0.0}, {0.5, 1.5, 0.0}, {3, 1.5, 0.0}}
    };
    glEnable(GL_AUTO_NORMAL);
    glEnable(GL_NORMALIZE);
    // 2D Bézier surface
    glMap2f(GL_MAP2_VERTEX_3, 0.0, 1.0, 3, 4, 0.0, 1.0, 12, 4, &ctrlpoints[0][0][0]);
    glEnable(GL_MAP2_VERTEX_3);
    glMapGrid2f(50, 0.0, 1.0, 50, 0.0, 1.0);
    glEvalMesh2(GL_FILL, 0, 50, 0, 50);

}
\end{minted}

首先我们定义了一个 4x4 的控制点数组,然后使用 \texttt{glMap2f} 函数将其映射到 3D 空间中,然后使用 \texttt{glMapGrid2f} 函数将其映射到网格上,最后使用 \texttt{glEvalMesh2} 函数绘制 Bézier 曲面。

效果如下图所示:

\begin{figure}[H]
    \centering
    \includegraphics[width=0.6\textwidth]{./figures/2024-11-28-23-48-42.png}
    \caption{Bézier 曲面}
\end{figure}

\section{Polygonal Mesh}
这里我们采用导入模型的方式来实现多边形网格。根据要求,我们首先从网络上找了车的模型,然后实现函数来导入模型。导入模型的代码如下:

\begin{minted}[    
    linenos,
    breaklines,
    bgcolor = cmdbg,
    fontsize=\footnotesize,
    frame=lines
]{cpp}
void loadOBJ(const char* filename) {
    std::vector<GLfloat> vertices;
    std::vector<GLuint> indices;
    std::ifstream file(filename);
    if (!file.is_open()) {
        std::cerr << "Failed to open file: " << filename << std::endl;
        return;
    }

    vertices.reserve(10000);
    indices.reserve(10000);

    std::string line;
    while (std::getline(file, line)) {
        if (line.substr(0, 2) == "v ") {
            std::istringstream s(line.substr(2));
            GLfloat x, y, z;
            s >> x; s >> y; s >> z;
            vertices.push_back(x);
            vertices.push_back(y);
            vertices.push_back(z);
        } else if (line.substr(0, 2) == "f ") {
            std::istringstream s(line.substr(2));
            std::string a, b, c;
            s >> a; s >> b; s >> c;
            indices.push_back(std::stoi(a.substr(0, a.find('/'))) - 1);
            indices.push_back(std::stoi(b.substr(0, b.find('/'))) - 1);
            indices.push_back(std::stoi(c.substr(0, c.find('/'))) - 1);
        }
    }
    file.close();

    glColor3f(0.8, 0.8, 0.0);
    glBegin(GL_TRIANGLES);
    for (size_t i = 0; i < indices.size(); i+=3) {
        glVertex3f(vertices[indices[i] * 3], vertices[indices[i] * 3 + 1], vertices[indices[i] * 3 + 2]);
        glVertex3f(vertices[indices[i+1] * 3], vertices[indices[i+1] * 3 + 1], vertices[indices[i+1] * 3 + 2]);
        glVertex3f(vertices[indices[i+2] * 3], vertices[indices[i+2] * 3 + 1], vertices[indices[i+2] * 3 + 2]);
    }
    glEnd();
}
\end{minted}

首先我们解析 .obj 文件,我们主要关注 v 和 f 两个部分,其中 v 表示顶点,f 表示面。f 是由点进行描述的,部分是有四边形描述的,对于这种我们要将其拆分为两个三角形,这个函数的实现我使用了 python。然后我们将其绘制出来。由于电脑性能缘故,我只能绘制相对简单的图案,效果如下:

\begin{figure}[H]
    \centering
    \includegraphics[width=0.6\textwidth]{./figures/2024-11-29-00-01-58.png}
    \caption{Polygonal Mesh}
\end{figure}

\section{L-System}
最后我们尝试了 L-System 来绘制车的形状。L-System 是一种递归的字符串替换系统,我们可以通过 L-System 来绘制各种各样的图案。这里我们使用 L-System 来绘制车的装饰形状。我们首先定义了一个 L-System,然后使用递归来绘制图案。代码如下:

\begin{minted}[    
    linenos,
    breaklines,
    bgcolor = cmdbg,
    fontsize=\footnotesize,
    frame=lines
]{cpp}
void drawBranch(int depth, float length){
    if (depth == 0) {
        glutSolidCube(length);
        return;
    }

    // 绘制当前段
    glutSolidCube(length);

    // 生成三个子块向外扩散
    for (int k = 0; k < 6; ++k) {
        glPushMatrix();
        glRotatef(k * 60.0, 0.0, 0.0, 1.0); // Rotate to create 3 sub-branches
        glTranslatef(length, 0.0, 0.0); // Translate to position the sub-branch
        drawBranch(depth - 1, length * 0.5); // 递归绘制子块
        glPopMatrix();
    }

    // 在 Y 轴方向上延伸
    glPushMatrix();
    glTranslatef(0.0, length, 0.0); // Translate along Y axis
    drawBranch(depth - 1, length * 0.5); // 递归绘制子块
    glPopMatrix();
}
\end{minted}

绘制时做了更多位移和旋转操作,使得图案在画布中的位置更加合理。效果如下:

\begin{figure}[H]
    \centering
    \includegraphics[width=0.6\textwidth]{./figures/2024-11-29-00-10-30.png}
    \caption{L-System}
\end{figure}

\section{最终效果}

本次实验还要求我们把建模得到的车子放入之前的太阳系中,为此,我对之前的模块进行了封装,在本次调用了这些模块,将车子放入太阳系中。效果如下:

\begin{figure}[H]
    \centering
    \includegraphics[width=0.6\textwidth]{./figures/2024-11-29-00-12-47.png}
    \caption{最终效果}
\end{figure}

\section{总结}
本次作业的最大收获是学会了如何导入已经建模好的模型,以及对模块的封装做了实践,方便之后更好的调用,为大作业做好基础。




\end{document}