%!TeX program = xelatex
\documentclass[12pt,hyperref,a4paper,UTF8]{ctexart}
\usepackage{zjureport}
\usepackage{tikz}
\usepackage{pgfplots}
\usepackage{tabularx}
\usepackage{float}
\definecolor{cmdbg}{rgb}{0.92,0.92,0.92}

%%-------------------------------正文开始---------------------------%%
\begin{document}

%%-----------------------封面--------------------%%
\cover


\thispagestyle{empty} % 首页不显示页码

%%--------------------------目录页------------------------%%
\newpage
\tableofcontents

%%------------------------正文页从这里开始-------------------%
\newpage
本次作业使用两种光照模型,漫反射与镜面反射,使用 opengl 的固定管线实现了漫反射和镜面反射。

\section{光照开启}
在 opengl 中默认是关闭光照的,首先要启用光照模型,然后根据固定管线,设置光源参数,本次实验要求的是点光源。另外由于编写 shader 需要高版本的 opengl,我在本次作业中使用的是老版本的固定管线,所以对于太阳的处理是用发光材质来模拟的。

\begin{minted}[    
    linenos,
    breaklines,
    bgcolor = cmdbg,
    fontsize=\footnotesize,
    frame=lines
]{cpp}
void setupLighting(GLfloat *lightPos, GLfloat *_diffuseLight = NULL, GLfloat *_specularLight = NULL, int _num = 0) {
    // 启用光照
    glEnable(GL_LIGHTING);
    int lightNum = 0;
    lightNum = GL_LIGHT0 + _num;
    // 启用光源
    glEnable(lightNum);
    // 设置光源位置
    glLightfv(lightNum, GL_POSITION, lightPos);

    // 设置漫反射光
    if(_diffuseLight != NULL){
        glLightfv(lightNum, GL_DIFFUSE, _diffuseLight);
    }
    
    // 设置镜面反射光
    if (_specularLight != NULL) {
        glLightfv(lightNum, GL_SPECULAR, _specularLight);
    }
}
\end{minted}

\section{漫反射和镜面反射}
根据固定管线的实现,我们提供光源参数即可,具体如下

\begin{minted}[    
    linenos,
    breaklines,
    bgcolor = cmdbg,
    fontsize=\footnotesize,
    frame=lines
]{cpp}
    setupLighting(light, NULL, specularLight, 0);
    setupLighting(light, diffuseLight, NULL, 1);
\end{minted}

下面是实现的结果

\begin{figure}[H]
    \centering
    \includegraphics[width=0.6\textwidth]{./figures/2024-12-04-14-39-24.png}
    \caption{漫反射}
\end{figure}

\begin{figure}[H]
    \centering
    \includegraphics[width=0.6\textwidth]{./figures/2024-12-04-14-40-04.png}
    \caption{镜面反射}
\end{figure}

\begin{figure}[H]
    \centering
    \includegraphics[width=0.6\textwidth]{./figures/2024-12-04-14-40-33.png}
    \caption{漫反射和镜面反射}
\end{figure}

在最后的效果图中,我们可以看到两种反射的效果都在其中,光照设置的效果还是不错的。

但是由于固定管线的限制,对于中心太阳的设置就比较简陋了

\begin{figure}[H]
    \centering
    \includegraphics[width=0.6\textwidth]{./figures/2024-12-04-14-42-05.png}
    \caption{太阳}
\end{figure}

\section{总结}

本次作业使用固定管线完成了光照设置,但是新老版本的opengl不兼容,会出现一些问题,本次作业使用简单材质模拟了太阳的光照效果,但是效果不是很好,在大作业中应该会改进效果。

\end{document}