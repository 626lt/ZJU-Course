%!TeX program = xelatex
\documentclass[12pt,hyperref,a4paper,UTF8]{ctexart}
\usepackage{zjureport}
\usepackage{tikz}
\usepackage{pgfplots}
\usepackage{tabularx}
\usepackage{float}
\usepackage{framed}
\usepackage{xcolor}
\usepackage{media9}

\definecolor{cmdbg}{rgb}{0.92,0.92,0.92}
\definecolor{shadecolor}{gray}{0.9}

%%-------------------------------正文开始---------------------------%%
\begin{document}

%%-----------------------封面--------------------%%
\cover


\thispagestyle{empty} % 首页不显示页码

%%--------------------------目录页------------------------%%
\newpage
\tableofcontents

%%------------------------正文页从这里开始-------------------%
\newpage
\section{需求分析}
首先分析需求

\begin{shaded*}
    \begin{enumerate}
        \item 2 suns,2+ planets,1+ satellite 
        \item Planets orbit around the sun 
        \item Satellites orbit around its planet 
        \item Trajectories are not co-planar 
        \item Navigation in the system (3D viewing) 
    \end{enumerate}
\end{shaded*}

要求 5 个星球,其中 2 个为太阳,2 个为行星,1 个为卫星。行星绕太阳公转,卫星绕行星公转,轨迹不在同一平面上。要求能够在系统中导航(3D 视图)。接着逐个分析如何实现。

\paragraph{2 suns,2+ planets,1+ satellite}
确定颜色,大小,绘制实心球即可。

\paragraph{orbit around}
这里要求绕某个中心旋转,实现这种动画通过每一帧旋转一定角度即可。通过角度变量随时间变化实现。

\paragraph{Trajectories are not co-planar}
要求轨迹不在同一平面上,就是要求在动的同时做一个旋转,绕另一个轴旋转一定角度即可。

\paragraph{Navigation in the system}
这里的要求是调整视角,可以通过调整 camera 的位置和朝向实现。

\section{代码实现}
\subsection{定义变量}
在上面的分析中,我们将具体的需求分解成了离散的元素,据此我们可以确定一些需要的变量。
\paragraph{常量}
星球的大小,颜色,轨道半径,旋转速度,轨道角度
\paragraph{变量}
摄影机的位置,朝向,鼠标灵敏度,是否旋转

\subsection{绘制图形}
需要绘制轨道、星球,分别用如下实现:
\begin{minted}[    
    linenos,
    breaklines,
    bgcolor = cmdbg,
    fontsize=\footnotesize,
    frame=lines
]{cpp}
void star(const float* color, const float* sphere)
{
    glColor3f(color[0], color[1], color[2]);
    glutSolidSphere(sphere[0], sphere[1], sphere[2]);
}

void orbit(const int radius)
{
    #ifdef ORBIT
    glBegin(GL_LINE_LOOP);
    glColor3f(colorOfOrbit[0],colorOfOrbit[1],colorOfOrbit[2]);
    for(int i = 0; i < 360; i++)
    {
        float theta = i * M_PI / 180;
        glVertex3f(radius * cos(theta), 0.0, radius * sin(theta));
    }
    glEnd();
    #endif
    return;
}
\end{minted}

确定颜色后使用 \texttt{glColor3f} 设置颜色,使用 \texttt{glutSolidSphere} 绘制实心球。轨道的绘制使用点绘制圆形。添加了是否需要绘制轨道的宏定义,可以在编译时选择是否绘制轨道。

\subsection{组合图形}
上面我们已经有了基本的图形元素,现在通过 \texttt{display} 来将这些图形排列与显示。在函数中,设置是否转动,摄影位置以及角度,然后绘制星球和轨道。

\begin{minted}[    
    linenos,
    breaklines,
    bgcolor = cmdbg,
    fontsize=\footnotesize,
    frame=lines
]{cpp}
void display(){
    glClear(GL_COLOR_BUFFER_BIT | GL_DEPTH_BUFFER_BIT); // 清除颜色缓冲区和深度缓冲区
    if(rotate) angle += 3; // angle 更新才能开始转动,这里的 3 也是代表基本转动速度
    glLoadIdentity();
    gluLookAt(eyePosition[0], eyePosition[1], eyePosition[2], 0.0, 0.0, 0.0, 0.0, 1.0, 0.0); // 设置摄像机位置
    // 设置摄像机角度,围绕x轴和y轴旋转
    glRotated(camera_angle_v, 1.0, 0.0, 0.0); 
    glRotated(camera_angle_h, 0.0, 1.0, 0.0);
    // solar system 1
    glPushMatrix();
    {
        glRotatef(angle * speedOfSun, 0.0, 1.0, 0.0); // 公转
        orbit(orbitOfSun); // 太阳轨道
        glTranslatef(orbitOfSun, 0.0, 0.0); // 太阳位置
        star(colorOfSun, sphereOfSun); // 太阳
        // planet 1
        glPushMatrix();
        {
            glRotatef(angleOfPlanet1, 0.0, 0.0, 1.0); // 轨道调整
            glRotatef(angle * speedOfPlanet1, 0.0, 1.0, 0.0); // 行星1公转
            orbit(orbitOfPlanet1); // 行星1轨道
            glTranslatef(orbitOfPlanet1, 0.0, 0.0); // 行星1位置
            star(colorOfPlanet1, sphereOfPlanet1); // 行星1
        }
        glPopMatrix();
    }
    glPopMatrix();
    // solar system 2
    ...

}
\end{minted}

上面展示了如何设置视角以及更新旋转,然后使用了 \texttt{glPushMatrix} 和 \texttt{glPopMatrix} 来保存和恢复状态,实现相对运动,其余的星球和卫星的绘制类似。

\subsection{视图设置}
在三维的图像显示中,视图的设置保证了我们能够看到我们想要的图像。调整视口、投影矩阵和模型视图矩阵,以适应新的窗口大小。

\begin{minted}[    
    linenos,
    breaklines,
    bgcolor = cmdbg,
    fontsize=\footnotesize,
    frame=lines
]{cpp}
void reshape(int w, int h) {
    glViewport(0, 0, (GLsizei)w, (GLsizei)h);
    glMatrixMode(GL_PROJECTION); // 设置投影矩阵
    glLoadIdentity();
    gluPerspective(60.0, (GLfloat)w / (GLfloat)h, 0.01f, 2000.0f);
    glMatrixMode(GL_MODELVIEW); // 设置模型视图矩阵
    glLoadIdentity();
}
\end{minted}

\subsection{控制设置}
为了实现在系统中导航,我们需要设置一些控制,这里使用键盘和鼠标来控制视角和旋转,接受键盘和鼠标事件,然后设置相应的变量。

\begin{minted}[    
    linenos,
    breaklines,
    bgcolor = cmdbg,
    fontsize=\footnotesize,
    frame=lines
]{cpp}
void onKeyboard(unsigned char key, int x, int y) {
    switch (key){
        case 'r':
            rotate = !rotate;
            break;
        case 27:
            exit(0);
            break;
        case 'w':
            eyePosition[1] += 10;
            break;
        case 's':
            eyePosition[1] -= 10;
            break;
        case 'a':
            eyePosition[0] -= 10;
            break;
        case 'd':
            eyePosition[0] += 10;
            break;
        default:
            break;
    }
}

void onSpecialKey(int key, int x, int y) {
    switch (key){
        case GLUT_KEY_UP:
            eyePosition[2] -= 10;
            break;
        case GLUT_KEY_DOWN:
            eyePosition[2] += 10;
            break;
        default:
            break;
    }
}
\end{minted}

键盘主要用于控制是否旋转,退出,上下左右移动,控制摄影机的位置。

\begin{minted}[    
    linenos,
    breaklines,
    bgcolor = cmdbg,
    fontsize=\footnotesize,
    frame=lines
]{cpp}
void mouseButton(int button, int state, int x, int y){
    if(button == GLUT_LEFT_BUTTON){
        if(state == GLUT_DOWN){
            Dragging = true;
            drag_x_origin = x;
            drag_y_origin = y;
        }
        else{
            Dragging = false;
        }
    }
}

void mouseMove(int x, int y){
    if(Dragging){
        camera_angle_h += (x - drag_x_origin) * mouse_sensitivity;
        camera_angle_v += (y - drag_y_origin) * mouse_sensitivity;
        drag_x_origin = x;
        drag_y_origin = y;
    }
    return;
}
\end{minted}

鼠标主要用于控制视角,左键按下时记录当前位置,移动时计算移动距离,更新视角。

\subsection{编译运行}
编译时需要链接 OpenGL 库,在 darwin 和 linux 下,链接的库不同,在程序和编译指令中都做了适配

\begin{minted}[    
    linenos,
    breaklines,
    bgcolor = cmdbg,
    fontsize=\footnotesize,
    frame=lines
]{makefile}
ifeq ($(UNAME_S), Linux)
	LDFLAGS := -lGL -lGLU -lglut
else ifeq ($(UNAME_S), Darwin)
	LDFLAGS := -L/System/Library/Frameworks -framework GLUT -framework OpenGL
else
    $(error Unsupported platform)
endif
\end{minted}

\begin{minted}
    [
        linenos,
        breaklines,
        bgcolor = cmdbg,
        fontsize=\footnotesize,
        frame=lines
    ]{cpp}
#if defined (__APPLE__)
    #define GL_SILENCE_DEPRECATION
    #include <GLUT/glut.h>
#elif defined(_WIN32) || defined(_WIN64) || defined(__linux__)
    #include <GL/glut.h>
#endif
\end{minted}

编译命令为 \texttt{make all},运行命令为 \texttt{make run}。另外提供了是否绘制轨道的选项,在编译时使用 \texttt{make all ORBIT=0} 即可。


\section{实现效果}

\begin{figure}[H]
    \centering
    \includegraphics[width=0.6\textwidth]{./figures/2024-10-27-16-20-54.png}
    \caption{太阳系模拟有轨道}
    \label{fig:solar_system}
\end{figure}

\begin{figure}[H]
    \centering
    \includegraphics[width=0.6\textwidth]{./figures/2024-10-29-16-03-16.png}
    \caption{太阳系模拟无轨道}
    \label{fig:solar_system_no_orbit}
\end{figure}

\end{document}