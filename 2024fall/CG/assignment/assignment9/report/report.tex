%!TeX program = xelatex
\documentclass[12pt,hyperref,a4paper,UTF8]{ctexart}
\usepackage{zjureport}
\usepackage{tikz}
\usepackage{pgfplots}
\usepackage{tabularx}
\usepackage{float}
\definecolor{cmdbg}{rgb}{0.92,0.92,0.92}

%%-------------------------------正文开始---------------------------%%
\begin{document}

%%-----------------------封面--------------------%%
\cover


\thispagestyle{empty} % 首页不显示页码

%%--------------------------目录页------------------------%%
\newpage
\tableofcontents

%%------------------------正文页从这里开始-------------------%
\newpage
本次作业为太阳系星球添加 texture ,bonus 部分是添加 skybox。

\section{太阳系星球添加 texture}

这一部分在固定管线的实现中,只需要 load 后绑定到对应的 uniform 变量即可。我们使用 \texttt{stb\_image.h} 来加载图片。然后把图片绑定到相应的 Texture ID 上。
并设置好对应的参数,包括S、T方向的重复方式,缩小和放大的方式等。

\begin{minted}[
    linenos,
    breaklines,
    bgcolor = cmdbg,
    fontsize=\footnotesize,
    frame=lines
]{cpp}
GLuint loadTexture(char const* path) {
    int width, height, nrChannels;
    unsigned char* data = stbi_load(path, &width, &height, &nrChannels, 0);

    GLuint textureID;
    glGenTextures(1, &textureID);
    glBindTexture(GL_TEXTURE_2D, textureID);

    if (data) {
        if (nrChannels == 3) {
            glTexImage2D(GL_TEXTURE_2D, 0, GL_RGB, width, height, 0, GL_RGB, GL_UNSIGNED_BYTE, data);
        } else if (nrChannels == 4) {
            glTexImage2D(GL_TEXTURE_2D, 0, GL_RGBA, width, height, 0, GL_RGBA, GL_UNSIGNED_BYTE, data);
        }
        glTexParameteri(GL_TEXTURE_2D, GL_TEXTURE_WRAP_S, GL_REPEAT);   
        glTexParameteri(GL_TEXTURE_2D, GL_TEXTURE_WRAP_T, GL_REPEAT);
        glTexParameteri(GL_TEXTURE_2D, GL_TEXTURE_MIN_FILTER, GL_LINEAR_MIPMAP_LINEAR);
        glTexParameteri(GL_TEXTURE_2D, GL_TEXTURE_MAG_FILTER, GL_LINEAR);

        glGenerateMipmap(GL_TEXTURE_2D);
    } else {
        std::cout << "Failed to load texture" << std::endl;
    }

    stbi_image_free(data);
    return textureID;  
}
\end{minted}

然后在绘制球体的时候,绑定对应的 Texture ID 即可。

\begin{minted}[
    linenos,
    breaklines,
    bgcolor = cmdbg,
    fontsize=\footnotesize,
    frame=lines
]{cpp}
void star(const float* color, const float* sphere, GLuint textureID) {
    // 设置颜色
    glColor3f(color[0], color[1], color[2]);

    // 启用纹理
    glEnable(GL_TEXTURE_2D);
    glBindTexture(GL_TEXTURE_2D, textureID);

    // 绘制带有纹理的球体
    GLUquadric* quad = gluNewQuadric();
    gluQuadricTexture(quad, GL_TRUE); // 启用纹理坐标生成
    gluQuadricNormals(quad, GLU_SMOOTH); // 设置平滑法线

    // 绘制球体
    gluSphere(quad, sphere[0], (int)sphere[1], (int)sphere[2]);

    // 清理
    gluDeleteQuadric(quad);
    glBindTexture(GL_TEXTURE_2D, 0);
    glDisable(GL_TEXTURE_2D);
}
\end{minted}

以下是添加 texture 后的效果图:

\begin{figure}[H]
    \centering
    \includegraphics[width=0.8\textwidth]{./figures/2024-12-05-21-35-56.png}
    \caption{太阳系星球添加 texture}
\end{figure}

这里的效果是添加了光照模型的结果,如果关闭光照那么效果如下

\begin{figure}[H]
    \centering
    \includegraphics[width=0.8\textwidth]{./figures/2024-12-05-21-38-08.png}
    \caption{太阳系星球添加 texture}
\end{figure}

尽管在 load texture 的时候已经设置了滤波的方式,但仍然有摩尔纹的情况。不知道什么原因。

\section{添加 skybox}

skybox 就是一个立方体,每个面都贴上了纹理。在绘制的时候,我们把 skybox 绘制在相机的位置,这样 skybox 就会在前景里面。

以下是实现的代码:

\begin{minted}[
    linenos,
    breaklines,
    bgcolor = cmdbg,
    fontsize=\footnotesize,
    frame=lines
]{cpp}
void drawSkybox(GLuint textureID[], float translateX, float translateY, float translateZ) {
    glEnable(GL_TEXTURE_2D);
    glPushMatrix();
    {   glClearColor(1.0, 1.0, 1.0, 1.0);
        glDisable(GL_LIGHTING);
        glTranslated(translateX, translateY, translateZ);
        glScaled(1.3, 1.3, 1.3);
        
        // 设置Skybox的大小
        float size = 900.0f;

        glBindTexture(GL_TEXTURE_2D, skyboxTextures[0]);
        glBegin(GL_QUADS);
        // 右
        glTexCoord2f(0.0f, 0.0f); glVertex3f(size, -size, -size);
        glTexCoord2f(1.0f, 0.0f); glVertex3f(size, -size,  size);
        glTexCoord2f(1.0f, 1.0f); glVertex3f(size,  size,  size);
        glTexCoord2f(0.0f, 1.0f); glVertex3f(size,  size, -size);
        glEnd();

        glBindTexture(GL_TEXTURE_2D, skyboxTextures[1]);
        glBegin(GL_QUADS);
        // 左
        glTexCoord2f(0.0f, 0.0f); glVertex3f(-size, -size,  size);
        glTexCoord2f(1.0f, 0.0f); glVertex3f(-size, -size, -size);
        glTexCoord2f(1.0f, 1.0f); glVertex3f(-size,  size, -size);
        glTexCoord2f(0.0f, 1.0f); glVertex3f(-size,  size,  size);
        glEnd();

        glBindTexture(GL_TEXTURE_2D, skyboxTextures[2]);
        glBegin(GL_QUADS);
        // 下
        glTexCoord2f(0.0f, 0.0f); glVertex3f(-size, -size, -size);
        glTexCoord2f(1.0f, 0.0f); glVertex3f( size, -size, -size);
        glTexCoord2f(1.0f, 1.0f); glVertex3f( size, -size,  size);
        glTexCoord2f(0.0f, 1.0f); glVertex3f(-size, -size,  size);
        glEnd();

        glBindTexture(GL_TEXTURE_2D, skyboxTextures[3]);
        glBegin(GL_QUADS);
        // 上
        glTexCoord2f(0.0f, 0.0f); glVertex3f(-size, size,  size);
        glTexCoord2f(1.0f, 0.0f); glVertex3f( size, size,  size);
        glTexCoord2f(1.0f, 1.0f); glVertex3f( size, size, -size);
        glTexCoord2f(0.0f, 1.0f); glVertex3f(-size, size, -size);
        glEnd();

        glBindTexture(GL_TEXTURE_2D, skyboxTextures[4]);
        glBegin(GL_QUADS);
        // 前
        glTexCoord2f(0.0f, 0.0f); glVertex3f(-size, -size, size);
        glTexCoord2f(1.0f, 0.0f); glVertex3f( size, -size, size);
        glTexCoord2f(1.0f, 1.0f); glVertex3f( size,  size, size);
        glTexCoord2f(0.0f, 1.0f); glVertex3f(-size,  size, size);
        glEnd();

        glBindTexture(GL_TEXTURE_2D, skyboxTextures[5]);
        glBegin(GL_QUADS);
        // 后
        glTexCoord2f(0.0f, 0.0f); glVertex3f( size, -size, -size);
        glTexCoord2f(1.0f, 0.0f); glVertex3f(-size, -size, -size);
        glTexCoord2f(1.0f, 1.0f); glVertex3f(-size,  size, -size);
        glTexCoord2f(0.0f, 1.0f); glVertex3f( size,  size, -size);
        glEnd();
    }
    glPopMatrix();

    glEnable(GL_LIGHTING);
    glDisable(GL_TEXTURE_2D);
}
\end{minted}

我发现我之前的视角设置有问题,导致 skybox 的效果不是很好,所以我修改了视角,使得 skybox 的效果成功成为不会随视角移动的背景。

\begin{figure}[H]
    \centering
    \includegraphics[width=0.8\textwidth]{./figures/2024-12-07-00-31-39.png}
    \caption{添加 skybox}   
\end{figure}

\section{总结}
在实现纹理贴图的时候较为顺利,但是在实现天空盒的时候,由于视角设置的问题,导致了 skybox 的效果不是很好。在修改视角后,skybox 的效果就很好了。同时,我发现在添加纹理的时候,有时候会出现摩尔纹,不知道是什么原因。

\end{document}